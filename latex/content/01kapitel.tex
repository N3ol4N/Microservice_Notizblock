%!TEX root = ../dokumentation.tex

\chapter{Einleitung}
\label{ch:Einleitung}
In der modernen Welt sind die Anforderungen an Software und deren Entwicklung sehr schnelllebig geworden. Die monolithische Software-Architektur, die früher in vielen Anwendungen verwendet wurde, ist für die Agile Softwareentwicklung und die damit verbundenen Anforderungen oft zu schwerfällig geworden.

Aus dieser Entwicklung heraus hat sich die Architektur der "Microservices" entwickelt. Diese sollen das Problem der ausgedehnten monolithischen Software in Angriff nehmen und durch viele kleine  verteilte Services ersetzen.

Einige der Vorteile dieser Aufteilung von Anwendungen in kleiner Bestandteile sind die schnelle Austauschbarkeit von einzelnen Softwarepaketen, die verkürzte Entwicklungszeit für die kleineren Services und die Modularität.

Das Ziel der hier vorliegenden Projektarbeit ist es einen kleinen Teil einer solchen Microserviceanwendung zu gestalten. Dabei kommt es darauf, an eine Anwendung zu erstellen, die den Ansprüchen von Microservices genügt und sich - wie oben beschrieben - auch in ein Netz aus mehreren Microservices integrieren lässt.