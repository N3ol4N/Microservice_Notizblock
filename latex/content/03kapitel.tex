%!TEX root = ../dokumentation.tex

\chapter{Datenbankverbindung}
\label{ch:Datenbankverbindung}
Die db.js legt die Verbindung zur Datenbank fest. Hier wird der URL-Pfad zum Mongo-Datenbank-Container festgelegt und exportiert. Somit kann diese URL von den anderen Teilen der Anwendung aufgerufen werden.
Zu beachten ist hier, dass der Pfad zum selbst bezeichneten Container festgelegt wird, nicht zu einer absoluten IP oder dem Localhost. \glqq  microservice-mongo\grqq{} beschreibt den Mongo-Db-Container der in der docker-compose.yaml erstellt wurde. \glqq  :27017\grqq{} ist hierbei der virtuelle Port der vom microservice-mongo-Container bereitgestellt wird. \glqq  Microservice\_Notizen\grqq{} ist in diesem Fall die konkrete Datenbank innerhalb der Mongo-DB.