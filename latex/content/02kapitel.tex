%!TEX root = ../dokumentation.tex

\chapter{Anforderungen}
\label{ch:Anforderungen}
Da im Zuge dieses Projektes eine Microservice-Anwendung erstellt werden soll, muss zuallererst festgelegt werden welche Anforderungen an das Projekt gestellt werden.

Diese Anwendung hat zum Ziel Notizen nachhaltig in eine dokumentenorientierte Datenbank abzuspeichern. Die Notizen sollen eingespeichert und wieder abgerufen werden können. Weiterhin sollen sie bearbeitet und gelöscht werden können.

Die Anwendung soll durch das Ansprechen von API's, sowie durch eine Benutzeroberfläche zur Verwaltung betrieben werden.

Für den Aufbau der Anwendung soll ein Microservice-Architektur umgesetzt werden.
Die Anwendung soll in Dockercontainern betrieben werden, was ebenfalls für die Datenbank gilt. Des Weiteren sollen die Container durch Docker-Compose orchestriert werden.

Zur Verdeutlichung dient \autoref{fig:Darstellung der Architektur}.
\begin{center}
\begin{figure}[h!]
\centering
\includegraphics[trim=1cm 1cm 1cm 1cm,height=.5\textwidth]{Schaubild.png}
\vspace{1pt}
\caption{Darstellung der Architektur}
\label{fig:Darstellung der Architektur}
\end{figure}
\end{center}
Zur konkreten Umsetzung sollen Node.js als Backend-Programmiersprache und eine Mongo-Datenbank für die Datenhaltung zum Einsatz kommen.

