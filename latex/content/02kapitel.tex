%!TEX root = ../dokumentation.tex

\chapter{Anforderungen}
\label{ch:Anforderungen}

Da im Zuge dieses Projektes eine Microservice-Anwendung erstellt werden soll, muss zu allererst festgelegt werden welche Anforderungen an das Projekt gestellt werden.

Diese Anwendung hat zum Ziel Notizen nachhaltig in eine dokumentenzentrierte Datenbank abzuspeichern. Die Notizen sollen eingespeichert und wieder abgerufen werden können. Weiterhin sollen sie bearbeitet und gelöscht werden können.

Die Anwendung soll durch Ansprachen von API's, sowie durch eine Benutzeroberfläche betrieben werden.

Für den Aufbau der Anwendung soll ein Microservice-Architektur umgesetzt werden.
Die Anwendung soll in Dockercontainern betrieben werden; gleiches gilt für die Datenbank. Außerdem sollen die Container durch Docker-Compose orchestriert werden.

Zur Verdeutlichung dient die folgende Abbildung

\begin{wrapfigure}{r}{0\textwidth}
\centering
\includegraphics[height=.5\textwidth]{Schaubild.png}
\vspace{3pt}
\caption{Schaubild\footnotemark}
\label{fig:blueant}
\end{wrapfigure}

Zur konkreten Umsetzung sollen Node.js als Backend-Programmiersprache und eine Mongo-Datenbank für die Datenhaltung zum Einsatz kommen.