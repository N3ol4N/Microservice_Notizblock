%!TEX root = ../dokumentation.tex

\chapter{Zusammenfassung}
\label{ch:Zusammenfassung}
Durch die vorliegende Anwendung ist es möglich eine Notizblock-Anwendung innerhalb von Dockercontainern auf jedem Rechner zu implementieren der Docker unterstützt.
Durch Docker-Compose muss dazu nur ein einziger Befehl ausgeführt werden.

Wahlweise ist es auch möglich die Anwendung als eigenständige Datenbank-Schnittstelle zu verwenden, indem man die \acs{API}'s, die im vorangegangenen \autoref{ch:Routes} beschrieben werden, direkt anspricht.

Node.js ermöglicht es, durch eine relativ überschaubare Codebasis, einen Server aufzusetzen.
Mongo-DB bietet als dokumentorientierte Datenbank eine ideale Grundlage für das Speichern von Notizen.
Durch Docker bzw. Dockercompose lässt sich die komplette Anwendung mit wenig Aufwand installieren und einsatzbereit machen.

Zusammenfassend lässt sich festhalten, dass die Kombination aus Node.js-Server, Mongo-DB und Docker gut geeignet ist, um schnell eine funktionierende Anwendung zu implementieren und durch die Möglichkeit Ports und Routes auch nach außen hin frei zugänglich zu machen. Dadurch lässt sich die Anwendung in jede beliebige Microservice-Architektur einbinden. 